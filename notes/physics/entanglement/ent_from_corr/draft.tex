\documentclass[11pt]{article}
\usepackage[margin=3cm]{geometry}
\geometry{margin=1in, headsep=0.25in}
\usepackage{amsmath}
\usepackage{amsthm}
\usepackage{amssymb}
\usepackage{physics}
\usepackage{graphicx}
\newtheorem{thm}{Theorem}
\newtheorem{lem}[thm]{Lemma}
\newtheorem{cor}{Corollary}[thm]
\theoremstyle{remark}
\newtheorem*{remark}{Remark}
\theoremstyle{definition}
\newtheorem{eg}{Example}[section]
\newtheorem{definition}{Definition}[section]
\usepackage{bbold}
\usepackage{hyperref}
\usepackage[usenames,dvipsnames]{xcolor}
\usepackage{tikz}
\usetikzlibrary{arrows.meta, automata, positioning, quotes, shapes}
\hypersetup{
	colorlinks=true,
	linkcolor=blue,
	filecolor=magenta,
	urlcolor=cyan,
}
\urlstyle{same}
\usepackage[sort&compress,numbers]{natbib}
\bibliographystyle{naturemag}
\usepackage{doi}
\newcommand{\todo}[1]{\textcolor{red}{TODO: #1}}
\title{From Correlation to Entanglement}
\author{Shi Feng}
\date{\today}
\begin{document}
\maketitle
In this writeup, I'm going to show the relation between reduced density matrix and the correlation function of fermion modes. 

First of all, I need to show that the correlation $A_{mn} = \expval{a_m^\dagger a_n}$ within a block of $L$ sites has nothing to do with its environment part of density matrix. This is explained in the appendix. As a result, the correlation matrix can be expressed as
\begin{equation}
	A_{mn} = \Tr(a_m^\dagger a_n \rho_L)
\end{equation}
Our goal is to invert this equation, i.e. to compute the RDM $\rho_L$ by correlation matrix  $A_{mn} = \expval{a_m^\dagger a_n}$. 

The matrix $A_{mn}$ is neccessarily Hermitian, since $A^\dagger \equiv A_{nm}^* = \expval{a_n^\dagger a_m}^* = \expval{(a_n^\dagger a_m)^\dagger} = \expval{a_m^\dagger a_n} = A_{mn}$. So $A_{mn}$ can be diagonalized by a unitary transformation $G = UAU^\dagger$:
\begin{equation}
	\begin{split}
		G_{pq} &= \sum_{m,n} U_{pm} A_{mn} U_{nq}^* = \sum_{m,n} U_{pm}\expval{a_m^\dagger a_n} U_{nq}^*\\
		       &= \expval{\left(\sum_{m} U_{pm} a_m^\dagger\right) \left(\sum_{n} a_n U_{nq}^*\right)  } \\
		       &\equiv \expval{g_p^\dagger g_q}\delta_{pq}
	\end{split}
\end{equation}
where the $\delta_{pq}$ comes from the fact that  $G_{pq}$ is diagonal. Now if we point to some element  $(m,n)$ of  $A_{mn}$, the element  $G_{mn}$ corresponding to the same index must satisfy
\begin{equation}
	G_{mn} = \sum_{m,n} U_{mm} \Tr(a_m^\dagger a_n \rho_L)U_{nn}^*  = \Tr(g_m^\dagger g_n \rho_L)
\end{equation}
It's readily to see that $g_m$ satisfies fermionic anti-commuatation:  $\{g_n, g_m^\dagger\} = \{\sum_{i}U_{ni} a_i,\; \sum_{j} a_j^\dagger U_{jm}^*  \} = \sum_{ij} U_{ni} U_{jm}^* \{a_i, a_j^\dagger\}  = \delta_{nm}$. This amounts to
\begin{equation}
	G_{mn} = \nu_{m} \delta_{mn} 
\end{equation}
where $\nu_m \equiv\expval{g_m^\dagger g_m}$ is the $m$-th eigen value of correlation matrix. It's worth pointing out that $g_m$  This implies that  $\rho_L$ is uncorrelated in the occupation number basis of $g_m^\dagger$. Hence  $\rho_L$ can be described by the following:
\begin{thm} \label{thm3}
	In a fermionic lattice model, the block (reduced) density matrix can be factorized under the basis that diagonalizes the correlation matrix $\expval{a_m^\dagger a_n}$:
\begin{equation}
	\rho_L = \varrho_1 \otimes \varrho_2 \otimes \ldots \otimes \varrho_L
\end{equation}
	where $\varrho_m$ is the single-mode density matrix corresponding to the $m$-th fermionic mode, and all $\varrho_m$ are neccessarily diagonal.  
\end{thm}

Let us represent $g_m$ and  $g_m^\dagger$ in their matrix representation. 
\begin{proof}
	We've shown that $G_{mn} = \Tr(a_m^\dagger a_n \rho_L) = 0$ if $m \neq n$. Since $g^\dagger$ creates fermion, we can denote the set of single-mode basis as $\{\ket{1}_m, \ket{0}_m\}$, the 1 fermion and 0-fermion respectively. Now we inspect modes $m$ and $n$, the relevant part of RDM is
	\begin{equation}
		\rho_L = \sum_{j,j'} c_{jj'}\ket{j}\bra{j'} 
	\end{equation}
	where $\ket{j} \in \{\ket{1_m1_n}, \ket{10},\ket{01}, \ket{00}\}$. The two-point correlation of modes $m$ and $n$ is
	\begin{equation}
		\begin{split}
		 	G_{mn} &= \Tr(a_m^\dagger a_n \rho_L) = \Tr(a_n \rho_L a_m^\dagger)\\
			       &= \sum_{i} \mel{i}{g_n \rho_L g_m^\dagger}{i} = \bra{00}g_n \rho_L g_m^\dagger \ket{00}\\
			       &= \sum_{jj'} c_{jj'}\braket{0_m 1_n}{j}\braket{j'}{1_m0_n}  \stackrel{!}{=} 0
		\end{split} 
	\end{equation}
	Now let's pick out matrix elements that do not annihilate the brakets, whose corresponding $c_{jj'}$ has to vanish.  These are:
	\[
		\ket{j}\bra{j'} \sim \ket{0_m 1_n} \bra{1_m 0_n} 
	\] 
	so the matrix element at $\ket{0_m}\bra{1_m}$ of single-mode RDM $\rho_m$, and the element at $\ket{1_n}\bra{0_n}$ of local RDM $\rho_n$ have to vanish. Also since $G_{mn}$ is diagonal, the other off-diagonal also vanishes. Therefore the only elements that survives are $c_{jj'}$ corresponding to
	 \[
		 \{\ket{0_m}\bra{0_m}, \ket{1_m}\bra{1_m}\} \otimes \{\ket{0_n}\bra{0_n}, \ket{1_n}\bra{1_n}\}
	\] 
	Hence all single-mode RDMs are neccessarily diagonal. 
\end{proof}
In the aforesaid basis, $g_m^\dagger$ and $g_m$ can be written as
\begin{equation}
	g_m = 
	\begin{pmatrix}
		0 & 0 \\ 1 & 0	
	\end{pmatrix},\;\;\;
	g_m^\dagger = 
	\begin{pmatrix}
		0 & 1 \\ 0 & 0 	
	\end{pmatrix}
\end{equation}
and note that the off-diagonal parts of single-mode RDM vanishes according to Theorem.\ref{thm3}, we can parameterized $\varrho_m$ by a undetermined variable  $\alpha_m$:
\begin{equation}
	\varrho_m = 
	\begin{pmatrix}
		\alpha_m & 0 \\ 0 & 1 - \alpha_m
	\end{pmatrix}
\end{equation}
which satisfies $\Tr(\varrho_m) = 1$.  
% Since $\rho_L$ can be factorized, fermions created by  $g_m^\dagger$ can be perceived as "localized" fermions in $m$-th state i.e. the amplitude of hoppoing $\expval{g_m^\dagger g_n} = 0,\; \forall m \neq n$, thus flavors of fermion are stationary. Hence the wavefunction can be though of as a product state of occupation numbers given by $g_m^\dagger g_m$.   Then we have \todo{}
% \begin{equation}
% 	\expval{g_m} = \Tr(g_m \rho_L) = Tr(g_m \varrho_m) = \beta_m = 0
% \end{equation}
Then eigen value of correlation matrix $\nu_m$ and that of single-mode RDM $\alpha_m$ can be related by:
\begin{equation}
	\nu_m = \Tr(g_m^\dagger g_m \varrho_m) = \Tr[
	\begin{pmatrix}
		1 & 0 \\ 0 & 0
	\end{pmatrix}
	\begin{pmatrix}
		\alpha_m & 0 \\ 0 & 1 - \alpha_m
	\end{pmatrix}
	] = \alpha_m
\end{equation}
where all the rest $\varrho_n$ with $n\neq m$ have only trival contribution to the trace. So we have
\begin{equation}
	\nu_m = \alpha_m
\end{equation}
Therefore the total entanglement of this block of $L$ sites is given by blocks $\varrho_m$ (Theorem.\ref{thm5}):
\begin{equation}
	S_L = \sum_{l=1}^L H_2(\nu_l) 
\end{equation}
where
\begin{equation}
	H_2(\nu_l) = - \nu_l \log \nu_l - (1 - \nu_l)\log(1 - \nu_l)
\end{equation}
is the binary entropy.



%\nocite{*}
%\printbibliography
%\bibliography{references.bib}
\end{document}
