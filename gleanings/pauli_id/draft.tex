\documentclass[11pt]{article}
\usepackage[margin=3cm]{geometry}
\geometry{margin=1in, headsep=0.25in}
\usepackage{amsmath}
\usepackage{amsthm}
\usepackage{amssymb}
\usepackage{physics}
\usepackage{graphicx}
\newtheorem{thm}{Theorem}
\newtheorem{lem}[thm]{Lemma}
\newtheorem{cor}{Corollary}[thm]
\theoremstyle{remark}
\newtheorem*{remark}{Remark}
\theoremstyle{definition}
\newtheorem{eg}{Example}[section]
\newtheorem{definition}{Definition}[section]
\usepackage{bbold}
\usepackage{hyperref}
\usepackage[usenames,dvipsnames]{xcolor}
\usepackage{tikz}
\usetikzlibrary{arrows.meta, automata, positioning, quotes, shapes}
\hypersetup{
	colorlinks=true,
	linkcolor=blue,
	filecolor=magenta,
	urlcolor=cyan,
}
\urlstyle{same}
\usepackage[sort&compress,numbers]{natbib}
\bibliographystyle{naturemag}
\usepackage{doi}
\newcommand{\todo}[1]{\textcolor{red}{TODO: #1}}
\title{An Interesting Proof of Pauli Identity}
\author{Shi Feng}
\date{}
\begin{document}
\maketitle
\begin{thm}
	Pauli Identity:	
\begin{equation}
	\sum_{a = 1,2,3}\sigma_{\alpha\beta}^a\sigma_{\gamma\delta}^a = 2\delta_{\alpha\delta}\delta_{\beta\gamma} -\delta_{\alpha\beta}\delta_{\gamma\delta} 
\end{equation}
with $\sigma$ the pauli matrices.
\end{thm}
\begin{proof}
The great orthoganality theorm can be used to prove this:
\begin{equation}
	\sum_{a = 1,2,3}\sigma_{\alpha\beta}^a\sigma_{\gamma\delta}^a = 2\delta_{\alpha\delta}\delta_{\beta\gamma} -\delta_{\alpha\beta}\delta_{\gamma\delta} 
\end{equation}
where $\sigma$ are pauli matrices. Note that 2-by-2 Pauli matrices belongs to irreducible representation of SU(2), so we can apply the theorem directly to Pauli group:
\begin{equation}
	G_{P} = \{ \pm I, \pm \sigma^x, \pm \sigma^y, \pm \sigma^z, \pm iI, \pm i\sigma^x, \pm i\sigma^y, \pm i\sigma^z \}
\end{equation}
Now we apply the GOT:
\begin{equation}
	\sum_{g\in G_P}D^\dagger(g)_{\alpha\beta}D(g)_{\gamma\delta} = 8 \delta_{\alpha\delta}\:\delta_{\gamma\beta} 
\end{equation}
Note that $\pm$ and $i$ has no effect on the L.H.S. due to the mutiplication with the complex conjuate, the summation $\sum_{g\in G_p}$ can be divided into 4 identical sums, of which we are interested the subset $P = \{I, \sigma^x, \sigma^y, \sigma^z\}$. This gives:
\begin{equation}
	\sum_{g\in P} D^\dagger(g)_{\alpha\beta}D(g)_{\gamma\delta} = 2\delta_{\alpha\delta}\delta_{\gamma\beta} 
\end{equation}
we can separate out the identity $I$ out of the sum, that is, whcih gives us  $\delta_{\alpha\beta}\delta_{\gamma\delta}$ on the L.H.S., and write the representation explicit by pauli matrices $\sigma$:
\begin{equation}
	\delta_{\alpha\beta}\delta_{\gamma\delta} + \sum_{a}\sigma_{\alpha\beta}^a \sigma_{\gamma\delta}^a = 2\delta_{\alpha\delta}\delta_{\gamma\beta}
\end{equation}
rearrange and we have the desired form:
\begin{equation}
	\sum_{a = 1,2,3}\sigma_{\alpha\beta}^a\sigma_{\gamma\delta}^a = 2\delta_{\alpha\delta}\delta_{\beta\gamma} -\delta_{\alpha\beta}\delta_{\gamma\delta} 
\end{equation}

\end{proof}


%\nocite{*}
%\printbibliography
%\bibliography{references.bib}
\end{document}
