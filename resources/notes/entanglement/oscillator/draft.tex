\documentclass[letterpaper,prb,superscriptaddress]{revtex4}
\usepackage{amsmath}
\usepackage{amssymb}
\usepackage{graphicx}
\usepackage[usenames,dvipsnames]{xcolor}
\usepackage{booktabs}
\usepackage{microtype}
\usepackage{hyperref}
\usepackage{amsthm}
\usepackage{physics}
\newtheorem{thm}{Theorem}
\newtheorem{lem}[thm]{Lemma}
\newtheorem{cor}{Corollary}[thm]
\theoremstyle{definition}
\newtheorem{definition}{Definition}[section]
\usepackage{bbold}
\urlstyle{same}
\bibliographystyle{apsrev}
\usepackage{doi}



\hypersetup{colorlinks=true,citecolor=Red,linkcolor=Blue,urlcolor=Black}
\AtBeginDocument{
	\heavyrulewidth=.08em
	\lightrulewidth=.05em
	\cmidrulewidth=.03em
	\belowrulesep=.65ex
	\belowbottomsep=0pt
	\aboverulesep=.4ex
	\abovetopsep=0pt
	\cmidrulesep=\doublerulesep
	\cmidrulekern=.5em
	\defaultaddspace=.5em
}

% START remove before publication
\definecolor{myred}{HTML}{FF4136}
\newcommand{\todo}[1]{\textcolor{myred}{FIXME: #1}}
% END remove before publication

\begin{document}

\title{Annotations of Peschel}
\author{Shi Feng}

\date{\today}
%\pacs{}

\begin{abstract}
TBW
\end{abstract}

\maketitle


\section{}
Consider a purely bosonic model, a chain of $L$ harmonic oscillator with frequency  $\omega_0$, coupled together by springs. It has a gap in the phonon spectrum and is a non-critical integrable system. The Hamiltonian reads
\begin{equation}
	H = \sum_{i=1}^L \left( -\frac{1}{2} \frac{\partial ^2}{\partial x_i^2} + \frac{1}{2} \omega_0^2 x_i^2 \right)   + \sum_{i=1}^{L-1} \frac{1}{2}\kappa(x_{i+1} - x_i)^2 
\end{equation}
Peschel parameterized it by $\omega_0 = 1 - \kappa$, so that if $\kappa = 0$ the Hamiltonian is digonal under boson occupation number, and there is no dispersion (only one mode $\omega_0$) and the system is gapped. If $\kappa \rightarrow 1$ (thus $\omega_0 \rightarrow 0$), there will only be acoustic phonon excitations and the system become gapless.  



\subsection{2 particle problem}
As the simplest example let us scrutinize the 2-particle problem. Its Hamiltonian reads
\begin{equation}
	H = \frac{1}{2} \frac{\partial^2 }{\partial x_1^2} + \frac{1}{2} \frac{\partial ^2}{\partial x_2^2} + \frac{1}{2}\omega_0^2 x_1^2 + \frac{1}{2} \omega_0^2 x_2^2 + \frac{\kappa}{2}(x_1 - x_2)^2
\end{equation}
We don't want off-diagonal terms like $x_1 - x_2$, so we do the following transformation:
\begin{equation}
	v = (x_1 + x_2)/\sqrt{2},\;\;u = (x_1 - x_2)/\sqrt{2}\;\;\iff\;\; x_1 =  (v + u)/\sqrt{2},\;\; x_2 = (v - u)/\sqrt{2}
\end{equation}
I like the factor of $\sqrt{2}$ because of its reciprocal symmetry (also the transformation belongs to $O(2)$ so that $\sum_{i} x_i^2$ remain the same form). Then the potential energy becomes
\begin{equation}
	\frac{1}{2} \omega_0 x_1^2 + \frac{1}{2}\omega_0 x_2^2 + \frac{\kappa}{2} (x_1 - x_2)^2 = \frac{1}{2}\omega_0 v^2 + \frac{1}{2} \omega_0 u^2 + \frac{\kappa}{4} u^2
\end{equation}
% Now one question worthy of asking is how this transformation affect momentum terms? 
% \begin{equation}
% 	\frac{\partial }{\partial x_1}=\frac{\partial u}{\partial x_1}\frac{\partial }{\partial u}+\frac{\partial v}{\partial x_1}\frac{\partial }{\partial v}=\frac{1}{\sqrt{2}}\left(\frac{\partial }{\partial u}+\frac{\partial }{\partial v}\right)
% \end{equation}
% \begin{equation}
% 	\frac{\partial }{\partial x_2}=\frac{\partial u}{\partial x_2}\frac{\partial }{\partial u}+\frac{\partial v}{\partial x_2}\frac{\partial }{\partial v}= \frac{1}{\sqrt{2}}\left(-\frac{\partial }{\partial u}+\frac{\partial }{\partial v}\right)
% \end{equation}
% so the second derivative gives
% \begin{equation}
% \begin{split}
% 	2\frac{\partial ^2}{\partial x_1^2} &= \sqrt{2}\frac{\partial }{\partial x_1} \left( \frac{\partial }{\partial u} + \frac{\partial }{\partial v}\right) = \left( \frac{\partial }{\partial u} + \frac{\partial }{\partial v}\right)\left( \frac{\partial }{\partial u} + \frac{\partial }{\partial v}\right) \\
% 					   &= \frac{\partial ^2}{\partial u^2} + 2 \frac{\partial ^2}{\partial u \partial v} + \frac{\partial ^2}{\partial v^2}
% \end{split}	
% \end{equation}
% \begin{equation}
% \begin{split}
% 	2\frac{\partial ^2}{\partial x_2^2} &= \sqrt{2}\frac{\partial }{\partial x_2} \left( -\frac{\partial }{\partial u} + \frac{\partial }{\partial v}\right) = \left( -\frac{\partial }{\partial u} + \frac{\partial }{\partial v}\right)\left( -\frac{\partial }{\partial u} + \frac{\partial }{\partial v}\right) \\
% 					   &= \frac{\partial ^2}{\partial u^2} - 2 \frac{\partial ^2}{\partial u \partial v} + \frac{\partial ^2}{\partial v^2}
% \end{split}	
% \end{equation}
% So the kinetic term after transformation is
% \begin{equation}
% 	\frac{1}{2}\frac{\partial ^2}{\partial x_1^2} + \frac{1}{2} \frac{\partial ^2}{\partial x_2^2} = \frac{1}{2}\frac{\partial ^2}{\partial u^2} + \frac{1}{2} \frac{\partial ^2}{\partial v^2}
% \end{equation}
% which is of the same form! Why is it happening? Do kinetic derivatives always remain the same form regardless of the diagonalization of potential terms in $x_i$? We can understand this intuitively by perceiving the Hamiltonian as a single oscillator in a 2D plane with $x_1$ and $x_2$ standing for the two axes. The symmetric matrix is orthogonally diagonalizable by $O(2)$, which is essentially a rotation of axes, thus shouldn't change the form of momentum. Then the original Hamiltonain is rotated into
\begin{equation}
	H = \frac{1}{2} \frac{\partial ^2}{\partial u^2} + \frac{1}{2}\left( \omega_0^2 + \frac{\kappa}{2} \right)u^2 + \frac{1}{2} \frac{\partial ^2}{\partial v^2} + \frac{1}{2}\omega_0^2 v^2 \equiv H_u + H_v
\end{equation}
which describes two de-coupled harmonic oscillators. Since $[H, H_u] = [H, H_v] = [H_u, H_v] = 0$, wavefunctions of two harmonic modes can be measured simultanously, and their corresponding wavefunctions become separable. The ground state of a 1D harmonic oscillator with angular frequency $\omega$ is
\begin{equation}
	\Psi(x) = \left( \frac{\omega}{\pi} \right)^{1/4}\exp(-\frac{\omega}{2} x^2) \exp(-i\frac{\omega}{2}t)
\end{equation}
therefore, if define $\Omega^2 \equiv (1/2)(\omega_0^2 + \kappa/2)$,  the joint wavefunction of normal modes is
\begin{equation}
	\Psi(u,v) = C \exp(-\frac{\Omega}{2}u^2-\frac{\omega_0}{2}v^2)
\end{equation}
where $C$ is a normalization constant. 
Next we are going to calculate the reduced density matrix of the state by tracing out one of the oscillators in the original coordiate. For example, let us trace out $x_1$ for the density matrix of  $x_2$:
\begin{equation}
	\begin{split}
		\rho_2(x_2, x_2') &= \int_{-\infty}^{\infty} dx_1 \Psi^*(x_1,x_2') \Psi(x_1,x_2) \\
			    &\propto \int_{-\infty}^{\infty} dx_1 \exp(-\frac{\Omega}{4}(x_1 - x_2')^2 - \frac{\omega_0}{4}(x_1 + x_2')^2) \exp(-\frac{\Omega}{4}(x_1 - x_2)^2 - \frac{\omega_0}{4}(x_1 + x_2)^2)\\
			    &= \exp{-\left(\frac{\omega_0 + \Omega}{4}\right)(x_2^2 + x_2'^2)}\int_{-\infty}^{\infty} dx_1 \exp{- \left( \frac{\omega_0 + \Omega}{2} \right)x_1^2 + \left[ \left( \frac{\Omega - \omega_0}{2} \right)(x_2 + x_2')\right]x_1} \\
			    &\propto  \exp{-\left(\frac{\omega_0 + \Omega}{4}\right)(x_2^2 + x_2'^2)} \exp{\frac{(\Omega - \omega_0)^2}{8(\omega_0 + \Omega)}(x_2 + x_2')^2}\\
			    &= \exp[-\gamma(x_2 + x_2')/2 + \beta x_2 x_2']
	\end{split}
\end{equation}
where
\[
	\beta = \frac{(\Omega - \omega_0)^2}{4(\omega_0 + \Omega)}, \;\;\;\;\gamma = \frac{\omega_0^2 + \Omega^2 + 6 \omega_0 \Omega}{4(\omega_0 + \Omega)}, \;\;\;\;\gamma - \beta = \frac{2\omega_0\Omega}{\omega_0 + \Omega}
.\] 
and the normalized reduced density matrix is
\begin{equation}
	\rho_2(x_2, x_2') = \sqrt{\frac{\gamma - \beta}{\pi}}\exp[-\frac{\gamma}{2}(x_2^2 + x_2'^2) + \beta x_2 x_2']
\end{equation}
To calculate von-Neumann entanglement entropy we need to solve the following eigenvalue problem:
\begin{equation}
	\int_{-\infty}^{\infty} dx' \rho_2(x,x') f_n(x') = p_n f_n(x)
\end{equation}
whereby the EE can obtained by $S = - \sum_{n} p_n \log p_n $. The solution can be guessed:
\begin{align}
	p_n &= (1 - \xi) \xi^n \\
	f_n(x) &= H_n(\alpha^{1/2}x) \exp(-\alpha x^2 /2)
\end{align}
where $H_n$ is the Hermit polynomial, $\alpha = (\gamma^2 - \beta^2)^{1/2}$, $\xi = \beta / (\gamma + \alpha)$. Then EE can be calculated by
\begin{equation}
	S = - \sum_{n} (1 - \xi) \xi^n \log(1 - \xi) \xi^n 
\end{equation}























\subsection{N particles}
Using PBC, the Hamiltonian can be written as
\begin{eqnarray}
H &=& \sum_{i=1}^N \left(-\frac{1}{2}\frac{\partial ^2}{\partial x_i^2} + \frac{1}{2}\omega_0^2 x^2_i +  \frac{\kappa}{2} (x_i-x_{i+1})^2 \right)\,, \qquad x_{N+1}=0\,,\\
&\equiv& \frac{1}{2}p^T Mp + \frac{1}{2}x^TKx\,,
\end{eqnarray}
where $M=I$ is diagonal and $K$ is a real symmetric $N\times N$ matrix with positive eigenvalues,
\begin{equation}
K=
\begin{pmatrix}
\kappa'& -\kappa & 0 & \cdots & 0 \\
-\kappa  & \kappa'& -\kappa & \ddots & \vdots \\
0 & -\kappa  & \ddots& \ddots & 0\\
\vdots & \ddots &\ddots & \kappa'&-\kappa \\
0&\cdots & 0 & -\kappa & \kappa'
\end{pmatrix}
\end{equation}
with $\kappa'_i = \omega_0^2+\kappa$. By choosing a basis which diagonalizes the matrix $K$, the hamiltonian can be express as the sum of uncoupled harmonic oscillators hamiltonian. That is
\begin{equation}
	X^T U^T \left( UKU^T \right) UX \equiv Y^TK_DY,\;\;\;\; \text{ with } U^TU = I
\end{equation}
where $K_D$ is a diagonal matrix whose elements are the square of angular frequencies $\omega_i^2$ of $i$-th normal modes. The resultant joint wavefunction takes the form
\begin{equation}
	\Psi(\mathbf{x}) \propto \exp{-\frac{Y^T \sqrt{K_D} T}{2}} = \exp{-\frac{X^T (U^T\sqrt{K_D}U)X}{2}} \equiv \exp{-\frac{X^T A X}{2}}
\end{equation}
where we defined the coupling matrix $A \equiv U^T \sqrt{K_D} U$ whose elements are the energies (characteriztic frequencies e.g. $\omega_{ij} x_ix_j$)  of bonds between oscillators. The normalized wavefunction then reads
\begin{equation}
	\Psi(\mathbf{x}) = \left( \frac{\det(A)}{\pi^N} \right)^{\frac{1}{4}}\exp{-\frac{X^T A X}{2}}
\end{equation}
Now that we have the phsyical intuition, let us trim and clarify some notations in order to be consistent with Peschel. We expand the exponential term, so that the wavefunction becomes
\begin{equation}
	\Psi(\mathbf{x}) = C \exp(-\frac{1}{2} \sum_{ij} A_{ij} x_i x_j )
\end{equation}
the coupling matrix $A$ can also be expanded by normal modes. Note that $\phi_q \in col(U), \; q = 1,\ldots,N$ are the set of normal basis, $A$ can be written as
\begin{equation}
	A_{ij} = \sum_{q=1}^N \omega_q \phi_q(i) \phi_q(j) 
\end{equation}
where $\omega_q \in \sqrt{K_D}$. Then the full density matrix is
\begin{equation}
\begin{split}
	\rho(\mathbf{x}, \mathbf{x}') &= \Psi(\mathbf{x})\Psi^*(\mathbf{x}') \propto \exp{-\frac{1}{2}\sum_{ij}  A_{ij} (x_i x_j + x_{i}' x_{j}') }\\
\end{split}
\end{equation}
To get the reduced density matrix $\rho_l$ of a single $l$-th oscillator we calculate the following:
\begin{equation}
	\rho_l(x_l, x_l') = \int \Bigl(\prod_{i\neq l} dx_i\Bigr)  \Psi(x_1,\ldots, x_l, \ldots, x_N) \Psi^*(x_1,\ldots, x_l', \ldots, x_N)
\end{equation}
where we set $x_i = x_i'$ if  $i \neq l$. With this restriction and noting that $A$ is symmetric, the full density matrix becomes
\begin{equation}
	\rho(x_1, \ldots x_l, \ldots, x_N,x_1, \ldots x_l', \ldots, x_N) = C\exp{-\sum_{i,j\neq l} A_{ij} x_i x_j - \sum_{j\neq l} A_{lj}x_j (x_l + x_l') - A_{ll} (x_l^2 + x_l'^2)  }
\end{equation}












%\section*{Acknowledgments}
%\bibliography{thesis}
\end{document}
